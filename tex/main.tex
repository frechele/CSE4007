\documentclass{article}

\usepackage{arxiv}
\usepackage{subfiles}
\usepackage[utf8]{inputenc} % allow utf-8 input
\usepackage[T1]{fontenc}    % use 8-bit T1 fonts
\usepackage{hyperref}       % hyperlinks
\usepackage{url}            % simple URL typesetting
\usepackage{booktabs}       % professional-quality tables
\usepackage{amsfonts}       % blackboard math symbols
\usepackage{nicefrac}       % compact symbols for 1/2, etc.
\usepackage{microtype}      % microtypography
\usepackage{cleveref}       % smart cross-referencing
\usepackage{lipsum}         % Can be removed after putting your text content
\usepackage{graphicx}
\usepackage{subcaption}
\usepackage{float}
\usepackage{natbib}
\usepackage{doi}
\usepackage[shortlabels]{enumitem}

%\title{Analysis of machine learning algorithms for hand-written digit data}
\title{Analysis of machine learning algorithms for \\small vision data}
\renewcommand{\shorttitle}{Analysis of machine learning algorithms for small vision data}

\author{Junyeong Park\\
		Department of Computer Science\\
		Hanyang University\\
		Seoul, South Korea\\
		\texttt{jyp10987@gmail.com}}
\date{}

\hypersetup{
	pdftitle={Untitled},
	pdfauthor={Junyeong Park},
}

\begin{document}
\maketitle

\begin{abstract}

The most recent machine learning trend is to build a large model based on a vast amount of data.
However, sometimes there may not be much data available.
In this situation, using a heavy model poses a risk of overfitting.
This paper presents which algorithms are most effective to use, especially in the field of computer vision, where there is little data.
We trained algorithms that are not burdensome to use with only a single CPU without special computing devices such as GPUs or TPUs.
And then we compared each model based on the time it took to be trained, the inference speed of the model, and translation invariance.
The experimental results demonstrate the convolutional neural network is the best model in general computer vision with a small dataset in terms of accuracy and robustness.

\end{abstract}

\section{Introduction}
\subfile{introduction}

\section{Experiments}
\subfile{experiments}

\section{Results}
\subfile{results}

\section{Conclusion}
\subfile{conclusion}

\end{document}